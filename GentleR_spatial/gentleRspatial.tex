
\documentclass{beamer}\usepackage{graphicx, color}
%% maxwidth is the original width if it is less than linewidth
%% otherwise use linewidth (to make sure the graphics do not exceed the margin)
\makeatletter
\def\maxwidth{ %
  \ifdim\Gin@nat@width>\linewidth
    \linewidth
  \else
    \Gin@nat@width
  \fi
}
\makeatother

\IfFileExists{upquote.sty}{\usepackage{upquote}}{}
\definecolor{fgcolor}{rgb}{0.2, 0.2, 0.2}
\newcommand{\hlnumber}[1]{\textcolor[rgb]{0,0,0}{#1}}%
\newcommand{\hlfunctioncall}[1]{\textcolor[rgb]{0.501960784313725,0,0.329411764705882}{\textbf{#1}}}%
\newcommand{\hlstring}[1]{\textcolor[rgb]{0.6,0.6,1}{#1}}%
\newcommand{\hlkeyword}[1]{\textcolor[rgb]{0,0,0}{\textbf{#1}}}%
\newcommand{\hlargument}[1]{\textcolor[rgb]{0.690196078431373,0.250980392156863,0.0196078431372549}{#1}}%
\newcommand{\hlcomment}[1]{\textcolor[rgb]{0.180392156862745,0.6,0.341176470588235}{#1}}%
\newcommand{\hlroxygencomment}[1]{\textcolor[rgb]{0.43921568627451,0.47843137254902,0.701960784313725}{#1}}%
\newcommand{\hlformalargs}[1]{\textcolor[rgb]{0.690196078431373,0.250980392156863,0.0196078431372549}{#1}}%
\newcommand{\hleqformalargs}[1]{\textcolor[rgb]{0.690196078431373,0.250980392156863,0.0196078431372549}{#1}}%
\newcommand{\hlassignement}[1]{\textcolor[rgb]{0,0,0}{\textbf{#1}}}%
\newcommand{\hlpackage}[1]{\textcolor[rgb]{0.588235294117647,0.709803921568627,0.145098039215686}{#1}}%
\newcommand{\hlslot}[1]{\textit{#1}}%
\newcommand{\hlsymbol}[1]{\textcolor[rgb]{0,0,0}{#1}}%
\newcommand{\hlprompt}[1]{\textcolor[rgb]{0.2,0.2,0.2}{#1}}%

\usepackage{framed}
\makeatletter
\newenvironment{kframe}{%
 \def\at@end@of@kframe{}%
 \ifinner\ifhmode%
  \def\at@end@of@kframe{\end{minipage}}%
  \begin{minipage}{\columnwidth}%
 \fi\fi%
 \def\FrameCommand##1{\hskip\@totalleftmargin \hskip-\fboxsep
 \colorbox{shadecolor}{##1}\hskip-\fboxsep
     % There is no \\@totalrightmargin, so:
     \hskip-\linewidth \hskip-\@totalleftmargin \hskip\columnwidth}%
 \MakeFramed {\advance\hsize-\width
   \@totalleftmargin\z@ \linewidth\hsize
   \@setminipage}}%
 {\par\unskip\endMakeFramed%
 \at@end@of@kframe}
\makeatother

\definecolor{shadecolor}{rgb}{.97, .97, .97}
\definecolor{messagecolor}{rgb}{0, 0, 0}
\definecolor{warningcolor}{rgb}{1, 0, 1}
\definecolor{errorcolor}{rgb}{1, 0, 0}
\newenvironment{knitrout}{}{} % an empty environment to be redefined in TeX

\usepackage{alltt}
\usepackage{url}
%  \let\oldfootnotesize\footnotesize
%  \renewcommand*{\footnotesize}{\oldfootnotesize\tiny}
 %%\renewcommand*{\footnotesize}{\oldfootnotesize\scriptsize}

\begin{document}




\title{R spatial, Michael Sumner}

\maketitle

\begin{frame}
\frametitle{Outline}
\tableofcontents
\end{frame}

\section{Spatial tools in R}
 \begin{frame}
  \frametitle{R tools for handling spatial data, gridded and vector}
 \begin{itemize} 
   \item traditionally a loose set of tools for spatial data
   \item prior to 2005 no serious organization outside of individual packages
   \item new classes (sp, raster, others) and tools for transforming between the variety of classes
   \item much more powerful but in some ways more complicated, since it all relies on extension packages
 \end{itemize} 
 \end{frame}

\begin{frame}
 \frametitle{Traditional R}
 \begin{itemize} 
   \item packages base, graphics, fields, spatial
   \item Data stored as point coordinates with attributes, data.frame, matrix, list
   \item vectors
   \item "atomic" vectors, character, complex, numeric, integer, logical, numeric, complex, raw, NULL
   \item "recursive" vectors, lists, expressions
   \item matrices and arrays, atomic vectors with dimension
   \item indexing with [ and [[
   \item Plotting engines base, grid, lattice
   \item plot, points, lines, polygon
   \item image, levelplot, persp, wireframe
 \end{itemize}
\end{frame}

\begin{frame}
 \frametitle{Current day tools}
  \begin{itemize}
    \item all the above still
    \item more advanced tools in spatstat, sp, raster;  uneasy marriage includes rgdal, maptools, rgeos, RNetCDF, ncdf/ncd4, maps/mapdata
  \end{itemize}
  
 Detailed and entertaining overview of Spatial in R:
 
\url{http://www.maths.lancs.ac.uk/~rowlings/Teaching/UseR2012/}
 
 
\end{frame}



\begin{frame}[fragile]
  \frametitle{Finding help with R}

\begin{itemize}
 \item 
\item there's an \textbf{R for Dummies} book
\item The R Inferno, gotchas for R users \url{http://www.burns-stat.com/documents/books/the-r-inferno/}
 \item R on \url{http://stackoverflow.com}
 \item do read and use the mailings lists \url{http://www.r-project.org/mail.html}
 \item Task Views on CRAN, for domain-specific materials \url{http://cran.csiro.au/web/views/}
 \item Contributed docs \url{http://cran.csiro.au/other-docs.html}
 \item re-read the FAQs \url{http://cran.csiro.au/faqs.html}
\item \textbf{sos} package 
\end{itemize}
\end{frame}
 
\end{document}
